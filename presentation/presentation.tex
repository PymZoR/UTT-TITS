\documentclass[t]{beamer}
\usepackage[utf8]{inputenc}
\usepackage[francais]{babel}
\usepackage{tikz,pgf}
\usepackage{wrapfig}
\usepackage{times}
\usepackage[absolute,overlay]{textpos}
\usepackage{url}
% \usepackage[texcoord,grid,gridunit=mm,gridcolor=red!10,subgridcolor=green!10]{eso-pic}

\definecolor{bleu}{rgb}{0.282, 0.439, 0.643}

\usecolortheme[named=bleu]{structure}
\useinnertheme[shadow]{rounded}

\hypersetup{pdfpagemode=FullScreen}

\setbeamersize{text margin left=0.6cm}
\setbeamersize{text margin right=0.4cm}
\setbeamersize{text margin left=.5cm, text margin right=.5cm}
\setbeamertemplate{navigation symbols}{}

\AtBeginSection[]{
    \begin{frame}
  	 \frametitle{\vspace*{-1cm} \begin{flushright} \ \ \vskip-0.25cm \end{flushright} }
        \vspace*{-0.75cm}
	    \begin{center}
            \centering
            \tableofcontents[currentsection]
	    \end{center}
    \end{frame}
}

\begin{document}
    \begin{frame}
        \pgfputat{\pgfxy(-0.505,0.685)}{\pgfbox[left,top]{\pgfimage[width=\paperwidth,height=\paperheight ]{./assets/UTT_garde_empty.pdf}}}
        \vspace*{2.5cm}

        \begin{center}
            \textcolor{white}{
                {\large Soutenance de TITS} \\
                {\footnotesize   Printemps 2015}\\
            \vspace*{0.5cm}
                {\footnotesize \textbf{Développement d'un robot autonome}}\\
            \vspace*{0.25cm}
                {\footnotesize Mousset Axel\\ Labate Aurélien}\\
          }
        \end{center}
    \end{frame}

\setbeamertemplate{background canvas}{
    \includegraphics[width=\paperwidth,height=\paperheight]{./assets/UTT_fond.pdf}}
    \addtobeamertemplate{footline}{
    \vspace{-0.5cm}
    \begin{flushright}
    \insertframenumber/\inserttotalframenumber
    \end{flushright}
}


\section{Introduction}
    \begin{frame}[c]
        \begin{center}
            \frametitle{\vspace*{-1.0cm} \begin{flushright} Introduction \ \ \vskip-0.25cm \end{flushright} }
            \vspace*{-0.75cm}
            \begin{itemize}
                \item Développement d'un robot autonome pour la coupe de France
                \item Classement: 54èmes sur plus de 180 équipes
                \item Casse mécanique du robot principal !
                \item Objectif: proposer des bases théoriques, algorithmiques et logicielles solides pour les années suivantes
            \end{itemize}
        \end{center}
    \end{frame}

    \begin{frame}[c]
        \frametitle{\vspace*{-1.0cm} \begin{flushright} Introduction \ \ \vskip-0.25cm \end{flushright} }
        La robotique est une discipline complexe, elle allie:
        \begin{center}
            \begin{itemize}
                \item Informatique
                \item Électronique
                \item Mécanique
                \item Sciences physiques, algorithmique..
            \end{itemize}
        \end{center}
        Dans cette présentation, on parlera d'\textbf{informatique} et de \textbf{réseau}.
    \end{frame}


\section{Architecture modulaire}
    \begin{frame}[c]
        \frametitle{\vspace*{-1.0cm} \begin{flushright} Architecture modulaire \ \ \vskip-0.25cm \end{flushright} }
        \vspace*{1.5cm}
        \begin{center}
            \textbf{Une architecture modulaire ?}
        \end{center}
        Inconveniants
        \begin{center}
            \begin{itemize}
                \item Plus cher
                \item Plus volumineux
                \item Plus de cables
            \end{itemize}
        \end{center}
        Avantages
        \begin{center}
            \begin{itemize}
                \item Réutilisable
                \item Facilité de séparation des tâches
                \item Changement facile de chaques modules
                \item Moins de problèmes en cas d'erreurs
            \end{itemize}
        \end{center}
    \end{frame}
    \begin{frame}[c]
        \frametitle{\vspace*{-1.0cm} \begin{flushright} Architecture modulaire \ \ \vskip-0.25cm \end{flushright} }
        \vspace*{1.5cm}
        \begin{center}
            \textbf{Le coeur}
        \end{center}
        \begin{center}
            \begin{itemize}
                \item Commande tous les autres modules
                \item Plus de puissance de calcul
                \item Language plus haut niveau
                \item \href{http://bot:8080/}{Interface de contrôle}
            \end{itemize}
        \end{center}

        Les autres modules :
        \begin{center}
            \begin{itemize}
                \item Moteur
                \item Pince
                \item Capteurs
            \end{itemize}
        \end{center}
    \end{frame}
    \begin{frame}[c]
        \frametitle{\vspace*{-1.0cm} \begin{flushright} Architecture modulaire \ \ \vskip-0.25cm \end{flushright} }
        \vspace*{1.5cm}
        \begin{center}
            \textbf{Module moteur}
        \end{center}
        \begin{center}
            \begin{itemize}
                \item Récupère les informations des roues codeuses
                \item Commande et asservis les moteurs
                \item Calcul la position et la trajectoire
            \end{itemize}
        \end{center}
    \end{frame}
    \begin{frame}[c]
        \frametitle{\vspace*{-1.0cm} \begin{flushright} Architecture modulaire \ \ \vskip-0.25cm \end{flushright} }
        \vspace*{1.5cm}
        \begin{center}
            \textbf{Module pince}
        \end{center}
        \begin{center}
            \begin{itemize}
                \item Génère les impulsions à fréquence optimale pour controller les moteurs
                \item Initialise la position des moteurs pas à pas
            \end{itemize}
        \end{center}
    \end{frame}
\section{Technologies de communication}
        \begin{frame}[c]
            \frametitle{\vspace*{-1.0cm} \begin{flushright} Technologies de communication \ \ \vskip-0.25cm \end{flushright} }
            \vspace*{0.5cm}
             \begin{textblock*}{1cm}(4cm,3cm)
                \textbf{UART}
            \end{textblock*}
            \begin{center}
                \begin{itemize}
                    \item Liaison full-duplex
                    \item Niveau logique: TTL
                    \item Connexion asynchrone
                    \item CRC basique: bit de parité
                    \item Débit (baudrate) variable
                    \item \textbf{Une interface par périphériques, et autant de fils}
                \end{itemize}
            \end{center}
        \end{frame}

        \begin{frame}[c]
            \frametitle{\vspace*{-1.0cm} \begin{flushright} Technologies de communication \ \ \vskip-0.25cm \end{flushright} }
            \vspace*{0.5cm}
             \begin{textblock*}{1cm}(4cm,3cm)
                \textbf{SPI}
            \end{textblock*}
            \begin{center}
                \begin{itemize}
                    \item Liaison full-duplex
                    \item Niveau logique: TTL
                    \item Connexion asynchrone
                    \item CRC basique: bit de parité
                    \item Débit (baudrate) variable
                    \item \textbf{Une interface par périphériques, et autant de fils}
                \end{itemize}
            \end{center}
        \end{frame}

        \begin{frame}[c]
            \frametitle{\vspace*{-1.0cm} \begin{flushright} Technologies de communication \ \ \vskip-0.25cm \end{flushright} }
            \vspace*{0.5cm}
             \begin{textblock*}{1cm}(4cm,3cm)
                \textbf{CAN}
            \end{textblock*}
            \begin{center}
                \begin{itemize}
                    \item Liaison full-duplex
                    \item Niveau logique: TTL
                    \item Connexion asynchrone
                    \item CRC basique: bit de parité
                    \item Débit (baudrate) variable
                    \item \textbf{Une interface par périphériques, et autant de fils}
                \end{itemize}
            \end{center}
        \end{frame}

        \begin{frame}[c]
            \frametitle{\vspace*{-1.0cm} \begin{flushright} Technologies de communication \ \ \vskip-0.25cm \end{flushright} }
            \vspace*{0.5cm}
             \begin{textblock*}{1cm}(4cm,3cm)
                \textbf{I2C}
            \end{textblock*}
            \begin{center}
                \begin{itemize}
                    \item Liaison full-duplex
                    \item Niveau logique: TTL
                    \item Connexion asynchrone
                    \item CRC basique: bit de parité
                    \item Débit (baudrate) variable
                    \item \textbf{Une interface par périphériques, et autant de fils}
                \end{itemize}
            \end{center}
        \end{frame}



\section{Conclusion}
    \begin{frame}[c]
        \frametitle{\vspace*{-1.0cm} \begin{flushright} Conclusion \ \ \vskip-0.25cm \end{flushright} }
        \vspace*{-0.75cm}
        \begin{itemize}
            \item Travail important sur les parties réutilisables
            \item Une documentation interne au club
            \item Le travail continu :
            \begin{itemize}
                \item Le bus can
                \item La modularité jusqu'au physique
                \item Des Arduino faites maison
                \item Moteur brushless
            \end{itemize}
            \item Un robot que nous pour les prochains évènements
        \end{itemize}

    \end{frame}

\section{Démonstration}
    \begin{frame}[c]
        \frametitle{\vspace*{-1.0cm} \begin{flushright} Démonstration \ \ \vskip-0.25cm \end{flushright} }
        \vspace*{-0.75cm}
        \begin{center}
            \href{./assets/cut.mp4}{Vidéo}
        \end{center}
    \end{frame}



\end{document}