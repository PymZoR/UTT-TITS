\chapter{Introduction}
    L’objectif d’une TITS est de réaliser un projet avec un aspect scientifique de recherche, et une mise en application technique en découlant. La robotique, en tant que discipline hybride réunit parfaitement ces deux aspects tout en découpant sa dimension technique en un large spectre de compétences différentes : électroniques, informatiques et mécaniques, sans parler de management et de gestion de projet.\\

    Nous avons donc choisis de mener à bien le développement d’un robot. Les objectifs à remplir pour ce robot étaient sa parfaite autonomie, sa capaciter à pouvoir se repérer et se déplacer, et la gestion d’une pince mono-axe. Ce projet s’inscrit dans une dynamique associative : le robot réalisé a participé à la coupe de France de robotique, qui est un concours scientifique à l’échelle nationale réunissant plus de 200 équipes, sur lequel nous avons finit 54èmes. De plus comme notre association participe tous les ans à la Coupe, nous avons aussi souhaité faire en sorte que notre travail soit réutilisable. Ce projet a donc impliqué plus de quatres autres personnes régulièrement, étudiants, professeurs, et techniciens de l’UTT. Il est donc important de noter que ce document décrit l’essentiel de notre travail, à savoir électronique et informatique, mais aussi le projet dans sa globalité car il nous parait pertinent de replacer les réalisations dans leurs contexte.