\chapter{Suivi du projet}
    \begin{tabular}{|p{3cm}|c|p{10cm}|}
        \hline
        Personnes présentes & Date & Objet et synthèse\\
        \hline
        \hline
        Aurélien Labate, Axel Mousset, Alexandre Vial & 18/03/15 & Délimitation du sujet de TITS\\
        \hline
        Aurélien Labate, Axel Mousset, Alexandre Vial & 25/03/15 & Début du développement de l'interface de débug\\
        \hline
        Aurélien Labate, Axel Mousset, Alexandre Vial & 1/04/15 & Recherches documentaires sur Odométrie, Asservissement\\
        \hline
        Aurélien Labate, Axel Mousset, Alexandre Vial & 8/04/15 & Suite des recherches documentaires, premières discussions sur les solutions trouvées. Premier jet des cartes électroniques.\\
        \hline
        Aurélien Labate, Axel Mousset, Alexandre Vial & 15/04/15 & Fin des recherches. Développement des outils de simulation d'asservissement.\\
        \hline
        Aurélien Labate, Axel Mousset, Alexandre Vial & 22/04/15 & Développement du traceur de courbes pour pouvoir interpréter les réponses moteur.\\
        \hline
        Aurélien Labate, Axel Mousset, Alexandre Vial & 29/04/15 & Impression des cartes électroniques. Assemblage sur la mécanique et premiers tests d'asservissements polaire.\\
        \hline
        Aurélien Labate, Axel Mousset, Alexandre Vial & 6/05/15 & Mise au point, harmonistation des différents modules et préparation du départ à la coupe de France.\\
        \hline
        Annulé & 13/05/15 & \textbf{Coupe de France de robotique}\\
        \hline
        Aurélien Labate, Axel Mousset, Alexandre Vial & 20/05/15 & Lol\\
        \hline
        Aurélien Labate, Axel Mousset, Alexandre Vial & 27/05/15 & Lol\\
        \hline
        Aurélien Labate, Axel Mousset, Alexandre Vial & 3/06/15 & Lol\\
        \hline
     \end{tabular}


\newpage
\chapter{Memento sur les influences des coefficients PID}