\chapter{Conclusion}
    La réalisation du robot a été menée à son terme dans les délais que nous nous étions imposé pour concourrir à la coupe de France. Néanmoins, une casse de nature mécanique a empêché le robot d'être oppérationnel dès le premier jour. Ainsi, le robot secondaire (développé par d'autres étudiants) a quand même pu marquer des points.\\

    Malgré cet accident, ce projet nous a donné satisfaction. C'est en effet un réel projet d'ingénierie, alliant autant compétences scientifiques, techniques et humaines pendant plus de six mois d'efforts motivés par l'envie de performer dans une compétition.\\

    Au niveau personnel, l'ensemble des connaissances acquises est très enrichissant. Il a fallu effectuer un large travail documentaire en amont sur des sujets comme l'automatisation, l'asservissement ou encore l'odométrie. C'est donc avec passion que nous nous sommes attachés à mettre en pratique ces nouvelles connaissances aux côtés de celles apprises en cours pour réaliser ce robot.\\

    A un niveau plus large, nous éspérons que ce travail pourra répondre à l'une des problématique majeure dans le monde associatif à l'UTT, à savoir la transmission des savoirs. Le projet a donc largement été penser pour devenir potentiellement de base de référence pour futures années du club de robotique de l'UTT.